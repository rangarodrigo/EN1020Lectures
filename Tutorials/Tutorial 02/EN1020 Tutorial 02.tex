% --------------------------------------------------------------
% This is all preamble stuff that you don't have to worry about.
% Head down to where it says "Start here"
% --------------------------------------------------------------

\documentclass[11pt]{article}
\usepackage[margin=1in]{geometry}
\usepackage{amsmath,amsthm,amssymb}
%\usepackage{multicol}
\usepackage{graphicx}
%\usepackage{fixltx2e}
%\usepackage{amsmath}

\usepackage{tikz}
\usepackage{pgfplots}
\usepackage{fourier}
\usepackage[inline]{enumitem}



\title{\Large Department of Electronic and Telecommunication Engineering\\University of Moratuwa\\Sri Lanka\\{\LARGE \bf \textsc{EN1060 Signals and Systems: Tutorial 02 \footnote{All the questions are from Oppenheim \emph{et al.} chapter 3.}}}}

\date{\vspace{-0.2in}\today}


\newcommand{\N}{\mathbb{N}}
\newcommand{\Z}{\mathbb{Z}}

\begin{document}



\maketitle
\noindent \tikz \draw (0,0) -- (\textwidth,0);

\begin{enumerate}
% Q01 Oppenheim et al. 3.21
\item A continuous-time periodic signal $x(t)$ is real valued and has a fundamental period $T=8$. The non-zero Fourier series coefficients for $x(t)$ are specified as
    \begin{equation*}
        a_1 = a^\ast_{-1} = j, \quad a_5 = a^\ast_{-5} = 2.
    \end{equation*}
    Express $x(t)$ in the form
    \begin{equation*}
        x(t) = \sum_{k=0}^{\infty}\cos(\omega_k t + \phi_k).
    \end{equation*}
    %\flushright [Oppenheim \emph{et al.} 3.21]\par


% Q02 Oppenheim et al. 3.22
\item Determine the Fourier series representation of the following signals:
    \begin{enumerate}
        \item Each $x(t)$ illustrated in Figure \ref{fi:fig1}.
        \item $x(t)$ periodic with period 2 and
        \begin{equation*}
            x(t) = e^{-t}\quad \text{for } \quad -1 < t < 1.
        \end{equation*}
        \item $x(t)$ periodic with period 4 and
        \begin{equation*}
            x(t) = \begin{cases}
                     \sin \pi t, & 0 \leq t \leq 2, \\
                     0, & 2 < t \leq 4.
                   \end{cases}
        \end{equation*}
    \end{enumerate}
    \begin{figure}
      \centering
      \begin{tikzpicture}[y=0.8cm]
	\begin{scope}
		\draw (-5, 0) -- (7,0) node[anchor=west] {\scriptsize $t$};
		\draw (0, -1.2) -- (0,2) node[anchor=south] {\scriptsize $x(t)$};	

		\foreach \t in {-4, -2, ..., 4}
		{
			\draw[thick] (\t, 0) -- ++(1, 1) --++(0, -2) -- ++(1,1);
		}
		
		\foreach \t in {-3, -1, ..., 5}
		{
			\node at (\t - 0.3, 0) [anchor=north] {\scriptsize $\t$};
		}
		
		\foreach \t in {-2, 2, 4}
		{
			\node at (\t-0.2 , 0) [anchor=south] {\scriptsize $\t$};
		}		
		
		\node at ( 0, 1) [anchor=east] {\scriptsize $1$};
	\end{scope}
	
	\begin{scope}[yshift=-4cm]
		\draw (-7, 0) -- (7,0) node[anchor=west] {\scriptsize $t$};
		\draw (0, -.2) -- (0,2) node[anchor=south] {\scriptsize $x(t)$};	

		\draw[thick] (-6,1) --++(1,0) -- ++(1, -1) (-2,0) -- ++(1,1) -- ++(2,0) -- ++(1,-1) (4, 0) -- ++(1,1) -- ++(1, 0);
		
		\foreach \t in {-3, -2, ..., 5}
		{
			\draw (\t, 0.1) -- ++(0, -0.1);
			\node at (\t - 0.3, 0) [anchor=north] {\scriptsize $\t$};
		}
		
		\node at ( 0, 1.2) [anchor=east] {\scriptsize $1$};
	\end{scope}
	
	\begin{scope}[scale=0.8, yshift=-9cm]
		\draw (-9, 0) -- (8,0) node[anchor=west] {\scriptsize $t$};
		\draw (0, -1.2) -- (0,2.5) node[anchor=south] {\scriptsize $x(t)$};	

		
		\foreach \t in {-8, -5, ..., 4}
		{
			\draw[thick] (\t, 0) -- ++(2, 2) --++(1, -2) ;
		}
		
		\foreach \t in {-8, -7, -6, -5, -4, -3, -2, 1, 2, 3, 4, 5, 6, 7}
		{
			\draw (\t, 0.1) -- ++(0, -0.1);
			\node at (\t, 0) [anchor=north] {\scriptsize $\t$};
		}
		
		\node at ( 0, 2) [anchor=east] {\scriptsize $2$};
	\end{scope}	
	
	\begin{scope}[yshift=-11cm]
		\draw (-5, 0) -- (7,0) node[anchor=west] {\scriptsize $t$};
		\draw (0, -2.2) -- (0,2) node[anchor=south] {\scriptsize $x(t)$};	

		\foreach \t in {-4, -2, ..., 6}
		{
			\draw[thick, -latex] (\t, 0) -- ++(0,1);
		}
		\foreach \t in {-3, -1, ..., 5}
		{
			\draw[thick, -latex] (\t, 0) -- ++(0,-2);
		}
				
		
		\foreach \t in {-3, -1, ..., 5}
		{
			\node at (\t , 0) [anchor=south] {\scriptsize $\t$};
		}
		
		\foreach \t in {-4, -2, 2, 4, 6}
		{
			\node at (\t , 0) [anchor=north] {\scriptsize $\t$};
		}		
		
		\node at ( 0, 1) [anchor=east] {\scriptsize $1$};
		\node at ( 0, -2) [anchor=east] {\scriptsize $-2$};		
	\end{scope}	
	
	
	\begin{scope}[yshift=-15.5cm]
		\draw (-8, 0) -- (7,0) node[anchor=west] {\scriptsize $t$};
		\draw (0, -1.2) -- (0,2) node[anchor=south] {\scriptsize $x(t)$};	

		\foreach \t in {-7.5, -1.5}
		{
			\draw[thick] (\t, 1) -| ++(0.5, -1)  ++(2, 0) |- ++(1, -1) -- ++(0,1) ++(2,0 )  |- ++(0.5, 1);
		}
		
		\draw[thick] (4.5, 1) -| ++(0.5, -1);
		
		\foreach \t in {-7, -6, -3, -2, -1, 3, 4, 5, 6}
		{
			\node at (\t, 0) [anchor=north] {\scriptsize $\t$};
		}
		
		\foreach \t in {-5, -4, 1, 2}
		{
			\node at (\t , 0) [anchor=south] {\scriptsize $\t$};
		}		
		
		\node at ( 0, 1) [anchor=east] {\scriptsize $1$};
	\end{scope}
		
		
	\begin{scope}[yshift=-19.5cm]
		\draw (-8, 0) -- (7,0) node[anchor=west] {\scriptsize $t$};
		\draw (0, -0.2) -- (0,2.2) node[anchor=south] {\scriptsize $x(t)$};	

		\foreach \t in {-6, -3, 0, 3}
		{
			\draw[thick] (\t, 0) |- ++(1, 2)  |- ++(1, -1) -- ++(0, -1);
		}
		
		\draw[thick] (4.5, 1) -| ++(0.5, -1);
		
		\foreach \t in {-7, -6, ..., -1, 1, 2, ..., 6}
		{
			\node at (\t, 0) [anchor=north] {\scriptsize $\t$};
		}
		
	
		
		\node at ( 0, 1) [anchor=east] {\scriptsize $1$};
		\node at ( 0, 2) [anchor=east] {\scriptsize $2$};
	\end{scope}
				
	
\end{tikzpicture} 
      \caption{Figure Q02}\label{fi:fig1}
    \end{figure}

    %\flushright [Oppenheim \emph{et al.} 3.22]

%Q03 Oppenheim et al. 3.23
\item In each of the following, we specify the Fourier series coefficients of a continuous-time signal that is periodic with period 4. Determine the signal $x(t)$ in each case.
    \begin{enumerate}
        \item $a_k = \begin{cases}0, & k =0,\\ (j)^k \frac{\sin k \pi/4}{k\pi}, & \text{otherwise}.        \end{cases}$
        \item $a_k = (-1)^k \frac{\sin k \pi/8}{2k\pi}$
        \item $a_k = \begin{cases} jk, & |k| < 3,\\ 0, & \text{otherwise}.        \end{cases}$
        \item $a_k = \begin{cases} 1, &  k \text{ even},\\ 2, & k \text{ odd}.        \end{cases}$
    \end{enumerate}


%Q04 Oppenheim et al. 3.24
\item Let
    \begin{equation*}
        x(t) = \begin{cases} t, &  0 \leq t \leq 1,\\ 2-t, & 1 \leq t \leq 2,        \end{cases}
    \end{equation*}
    be a periodic signal with fundamental period $T=2$ and Fourier coefficients $a_k$.
    \begin{enumerate}
        \item Determine the value of $a_0$.
        \item Determine the Fourier series representation of $dx(t)/dt$.
        \item Use this result and the differentiation property of the continuous-time Fourier series to help determine the Fourier series coefficients of $x(t)$.
    \end{enumerate}


%Q05 Oppenheim et al. 3.25
\item Consider the following three continuous-time signals with a fundamental period of $T=1/2$:
    \begin{eqnarray*}
    % \nonumber % Remove numbering (before each equation)
      x(t) &=& \cos(4\pi t), \\
      y(t) &=& \sin(4\pi t), \\
      z(t) &=& x(t)y(t).
    \end{eqnarray*}

    \begin{enumerate}
        \item Determine the Fourier series coefficients of $x(t)$.
        \item Determine the Fourier series coefficients of $y(t)$.
        \item \label{qu:5c} Use these results along with the multiplication property of the continuous-time Fourier series to determine the Fourier series coefficients of $z(t)$.
        \item Determine the Fourier series coefficients of $z(t)$ through direct expansion of $z(t)$ in trigonometric from, and compare the result with that of part \ref{qu:5c}.
    \end{enumerate}


%Q06 Oppenheim et al. 3.26
\item Let $x(t)$ be a periodic signal whose Fourier series coefficients are
    \begin{equation*}
        a_k = \begin{cases} 2, &  k=0 ,\\ j\left(\frac{1}{2}\right)^{|k|}, & \text{otherwise}.       \end{cases}
    \end{equation*}
    Use Fourier series properties to answer the Following questions:
    \begin{enumerate}
        \item Is $x(t)$ real?
        \item Is $x(t)$ even?
        \item Is $dx(t)/dt$ even?
    \end{enumerate}

%Q07 Oppenheim et al. 3.40
\item Let $x(t)$ be a periodic signal with fundamental period $T$ and Fourier series coefficients $a_k$. Derive the Fourier series coefficients of each of the following signals in terms of $a_k$:
    \begin{enumerate}
        \item $x(t-t_0) + x(t+t_0)$
        \item $\mathfrak{Ev}\left\{x(t)\right\}$
        \item $\mathfrak{Re}\left\{x(t)\right\}$
        \item $\dfrac{d^2 x(t)}{dt^2}$
        \item $x(3t -1)$ [for this part, first determine the period of $x(3t -1)$]
    \end{enumerate}


%Q08 Oppenheim et al. 3.41
\item Suppose we are given the following information about a continuous-time periodic signal with period 3 and Fourier coefficients $a_k$.
    \begin{enumerate}
      \item $a_k = a_{k+2}$.
      \item $a_k = a_{-k}$.
      \item $\int_{-0.5}^{0.5}x(t)dt = 1$.
      \item $\int_{0.5}^{1.5}x(t)dt = 2$.
    \end{enumerate}
    Determine $x(t)$.

%Q09 Oppenheim et al. 3.42
\item Let $x(t)$ be a real-valued signal with fundamental period $T$ and Fourier series coefficients $a_k$.
    \begin{enumerate}
      \item Show that $a_k = a^\ast_{-k}$ and $a_0$ must be real.
      \item Show that if $x(t)$ is even, then its Fourier series coefficients must be real and even.
      \item Show that if $x(t)$ is odd, then its Fourier series coefficients are imaginary and odd and $a_0 = 0$.
      \item Show that the Fourier series coefficients of the even part of $x(t)$ are equal to $\mathfrak{Re}\left\{a_k\right\}$.
      \item Show that the Fourier series coefficients of the odd part of $x(t)$ are equal to $j\mathfrak{Im}\left\{a_k\right\}$.
    \end{enumerate}

%Q10 Oppenheim et al. 3.45
\item Let $x(t)$ be a real periodic signal with Fourier series representation given in the sine-cosine form
    \begin{equation}\label{eq:sinecosine}
        x(t) = a_0 + 2\sum_{k=1}^{\infty}[B_k \cos k\omega_0 t - C_k \sin k\omega_0 t].
    \end{equation}
    \begin{enumerate}
      \item Find the exponential Fourier series representation of the even and odd parts of $x(t)$, that is, find the coefficients $\alpha_k$ and $\beta_k$ in terms of the coefficients in eq. \ref{eq:sinecosine} so that
          \begin{eqnarray*}
          % \nonumber % Remove numbering (before each equation)
            \mathfrak{Ev}\left\{x(t)\right\} &=& \sum_{k=-\infty}^{\infty} \alpha_k e^{jk\omega_0 t},\\
            \mathfrak{Od}\left\{x(t)\right\} &=&  \sum_{k=-\infty}^{\infty} \beta_k e^{jk\omega_0 t}.\\
          \end{eqnarray*}
      \item What is the relationship between $\alpha_k$ and $\alpha_{-k}$? What is the relationship between $\beta_k$ and $\beta_{-k}$?
    \end{enumerate}
\end{enumerate}
\end{document} 