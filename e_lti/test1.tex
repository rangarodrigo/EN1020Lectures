\begin{frame}{}

    {\bf Example}: Show that the complex exponential $e^{st}$, where $s$ is a complex variable\footnote{$s= \sigma + j\omega$}, is an eigenfunction of the LTI system with impulse response $h(t)$. Find en expression for the eigenvalue $H(s)$. Hence, show that $e^{j\omega t}$ is an eigenfunction of the LTI system with impulse response $h(t)$ and find the eigenvalue $H(j\omega)$.
    % \tikz[draw=black] \draw (0,0) -- (\textwidth,0);

    \pause
    \mode<beamer>
    {
        \begin{columns}[t]
            \begin{column}[t]{0.5\textwidth}
                \begin{align*}
                    y(t) &= h(t) * x(t) \\
                    y(t) &= \int_{-\infty}^{\infty} h(\tau) x(t - \tau) d\tau \\
                    y(t) &= \int_{-\infty}^{\infty} h(\tau) e^{s(t - \tau)} d\tau \\
                \end{align*} 
                Noting that $ e^{s(t - \tau)} = e^{st} e^{-s\tau}$ and $e^{st}$ is a constant with respect to $\tau$,           
            \end{column}
            \begin{column}[t]{0.5\textwidth}
                \begin{align*}
                    y(t) &= e^{st} \int_{-\infty}^{\infty} h(\tau) e^{-s\tau} d\tau \\
                    y(t) &= e^{st} H(s)  \\
                    H(s) &= \int_{-\infty}^{\infty} h(\tau) e^{-s\tau} d\tau
                \end{align*} 
                \pause
                Substituting $s = j\omega$, we have $y(t) = e^{j\omega t} H(j\omega)$.
                \begin{align*}
                    H(j\omega) &= \int_{-\infty}^{\infty} h(\tau) e^{-j\omega\tau} d\tau \\
                \end{align*}           
            \end{column} 
        \end{columns}
    }
\end{frame}

\begin{frame}

    {\bf Example}: Consider an LTI system for which the input and the output are related by 
    \begin{equation*}
        y(t) = x(t-3),
    \end{equation*}
    \begin{enumerate}
        \item If the input is the exponential signal $x(t) = e^{j2t}$, find the output $y(t)$ and the eigenvalue $H(s)$.
        \item Sate $h(t)$ and, hence, find $H(j\omega)$.
    \end{enumerate}
    

    \pause
    \mode<beamer>
    {
        \begin{columns}[t]
            \begin{column}[t]{0.5\textwidth}
                \begin{align*}
                    y(t) &= x(t-3)\\
                    &= e^{j2(t-3)} \\
                    &= e^{j2t} e^{-j6}  = e^{-j6} x(t)\\
                \end{align*} 
                As $y(t) = H(s)e^{st}$, we have $H(s) = e^{-j6}$.\\   
                \pause
                \begin{equation*}
                    h(t) = \delta(t-3)\; \text{by substituting } \delta(t) \text{ for } x(t).     
                \end{equation*}                   
            \end{column}
            \pause
            \begin{column}[t]{0.5\textwidth}   
                \begin{align*}
                    H(s) &= \int_{-\infty}^{\infty} h(\tau) e^{-s\tau} d\tau \\
                    &= \int_{-\infty}^{\infty} \delta(\tau-3) e^{-s\tau} d\tau \\
                    &= e^{-s3} \int_{-\infty}^{\infty} \delta(\tau) e^{-s\tau} d\tau \\
                    &= e^{-s3} \\
                \end{align*}
                So, $H(j\omega) = e^{-j3\omega}$. Specifically, $H(j2t) = e^{-j6}$       
            \end{column} 
        \end{columns}
    }
\end{frame}