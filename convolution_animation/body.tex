\begin{frame}{Convolution Integral and Sum}
    \begin{columns}
        \begin{column}{0.5\textwidth}
            The convolution of the signal $x(t)$ and $h(t)$ is given by
            \begin{equation*}
                \boxed{y(t) = \int_{-\infty}^{\infty} x(\tau)h(t - \tau)d\tau,}
            \end{equation*}
            which is referred to as the \alert{convolution integral} or the \alert{superposition integral}. This  corresponds to the representation of a continuous-time LTI system in terms of its response to a unit impulse.
            \begin{equation*}
                y(t) = x(t) \ast h(t).
            \end{equation*}                   
        \end{column}
        \begin{column}{0.5\textwidth}
            The convolution of the sequence $x[n]$ and $h[n]$ is given by
            \begin{equation}\label{eq:convolution_sum}
                \boxed{y[n] = \sum_{k=-\infty}^{\infty}x[k]h[n-k],}
            \end{equation}
            which is referred to as the \alert{convolution sum} or \alert{superposition sum}. This corresponds to the representation of a discrete-time LTI system in terms of its response to a unit impulse (sample), which we represent symbolically as
            \begin{equation}\label{eq:convolution_symbol}
                y[n] = x[n]\ast h[n].
            \end{equation}          
        \end{column}               
    \end{columns}
    \vspace*{1cm}
    \noindent The characteristics of an LTI system are completely determined  by its impulse response.      
\end{frame}

\begin{frame}{Examples for the Animation}
    \begin{enumerate}
        \item $y(t) = x(t) \ast \delta(t)$
        \item $y(t) = x(t) \ast \delta(t - 2)$
        \item $y(t) = x(t) \ast [\delta(t) + \delta(t - 2)]$
        \item $y(t) = x(t) \ast [u(t) - u(t-1)]$
    \end{enumerate}
\end{frame}

\begin{frame}{Example 4}
    \begin{enumerate}
        \item $y(t) = x(t) \ast \delta(t)$
        \item $y(t) = x(t) \ast \delta(t - 2)$
        \item $y(t) = x(t) \ast [\delta(t) + \delta(t - 2)]$
        \item $y(t) = x(t) \ast [u(t) - u(t-1)]$
    \end{enumerate}

    \pgfplotsset{every axis/.append style={font=\scriptsize}}
\begin{tikzpicture}[scale=0.6]
\begin{axis}[
	name=axis1,
       width=8cm,
       height=3cm,	
	axis y line=middle,
	axis x line=bottom,
	ymin=0,ymax=2.2,
	xmin=-4,xmax=4,	
	xlabel=$t$,
	ylabel= $h(t)$,	
	xtick={0, 1},
	xticklabels={0, $1$},
	ytick={1},
	every axis x label/.style={at={(ticklabel* cs:1.05)},    anchor=west},
	every axis y label/.style={at={(ticklabel* cs:1.05)},    anchor=west},	
	clip=false,
]
	\addplot [thick, EntcBlue] coordinates {(-3,0) (0,0) (0,1) (1,1) (1,0) (4,0)};
\end{axis}
\pause
\def\tstart{0}
\begin{axis}[
	name=axis2,
	at={($(axis1.south east)+(0,-0.8cm)$)},anchor=north east,
       width=8cm,
       height=3cm,	
	axis y line=middle,
	axis x line=bottom,
	ymin=0,ymax=2.2,
	xmin=-4,xmax=4,	
	xlabel=$\tau$,
	ylabel= $h(t-\tau)$,	
	xtick={-3.5, -0.5, 0},
	xticklabels={$t-3$, $t$, 0},
	ytick={1},
	yticklabels={$1$},
	every axis x label/.style={at={(ticklabel* cs:1.05)},    anchor=west},
	every axis y label/.style={at={(ticklabel* cs:1.05)},    anchor=west},	
	clip=false,
]
	\addplot [dashed, EntcBlue] coordinates {(-3,0) (0,0) (0,1) (1,1) (1,0) (4,0)};
	\addplot [thick, EntcRed] coordinates {(\tstart-3.5,0) (\tstart-2.5,2) (\tstart-0.5,0) (\tstart-0.5, 0) (\tstart + 4,0)};

	\node at (axis cs:3,0.8) {$t<0$};	
\end{axis}
\pause
\def\tstart{0.5}
\begin{axis}[
	name=axis3,
	at={($(axis2.south east)+(0,-0.8cm)$)},anchor=north east,
       width=8cm,
       height=3cm,	
	axis y line=middle,
	axis x line=bottom,
	ymin=0,ymax=2.2,
	xmin=-4,xmax=4,	
	xlabel=$\tau$,
	ylabel= $h(t-\tau)$,	
	xtick={-1.5, 0.5, 0},
	xticklabels={$t-2T$, $t$, 0},
	ytick={1, 2},
	yticklabels={$1$, $2$},
	every axis x label/.style={at={(ticklabel* cs:1.05)},    anchor=west},
	every axis y label/.style={at={(ticklabel* cs:1.05)},    anchor=west},	
	clip=false,
]
	\addplot [dashed, EntcBlue] coordinates {(-3,0) (0,0) (0,1) (1,1) (1,0) (4,0)};
	\addplot [thick, EntcRed] coordinates {(\tstart-3.5,0) (\tstart-2.5,2) (\tstart-0.5,0) (\tstart-0.5, 0) (\tstart + 4,0)};
	\node at (axis cs:3,0.8) {$0<t<T$};	
\end{axis}
\pause
\begin{axis}[
	name=axis4,
	at={($(axis3.south east)+(0,-0.8cm)$)},anchor=north east,
       width=8cm,
       height=3cm,	
	axis y line=middle,
	axis x line=bottom,
	ymin=0,ymax=1.2,
	xmin=-3,xmax=4,	
	xlabel=$\tau$,
	ylabel= $h(t-\tau)$,	
	xtick={-.5, 1.5},
	xticklabels={$t-2T$, $t$},
	ytick={1},
	yticklabels=\empty,
	every axis x label/.style={at={(ticklabel* cs:1.05)},    anchor=west},
	every axis y label/.style={at={(ticklabel* cs:1.05)},    anchor=west},	
	clip=false,
]
	\addplot [dashed, EntcBlue] coordinates {(-3,0) (0,0) (0,1) (1,1) (1,0) (4,0)};
	\addplot [thick, EntcRed] coordinates {(-3,0) (-.5,0) (-.5,1) (1.5, 0) (4,0)};
	\node at (axis cs:3,0.8) {$T<t<2T$};	
\end{axis}
\pause
\begin{axis}[
	name=axis5,
	at={($(axis4.south east)+(0,-0.8cm)$)},anchor=north east,
       width=8cm,
       height=3cm,	
	axis y line=middle,
	axis x line=bottom,
	ymin=0,ymax=1.2,
	xmin=-3,xmax=4,	
	xlabel=$\tau$,
	ylabel= $h(t-\tau)$,	
	xtick={.5, 2.5},
	xticklabels={$t-2T$, $t$},
	ytick={1},
	yticklabels={$2T$},
	every axis x label/.style={at={(ticklabel* cs:1.05)},    anchor=west},
	every axis y label/.style={at={(ticklabel* cs:1.05)},    anchor=west},	
	clip=false,
]
	\addplot [dashed, EntcBlue] coordinates {(-3,0) (0,0) (0,1) (1,1) (1,0) (4,0)};
	\addplot [thick, EntcRed] coordinates {(-3,0) (.5,0) (.5,1) (2.5, 0) (4,0)};
	\node at (axis cs:3,0.8) {$2T<t<3T$};	
\end{axis}
\pause
\begin{axis}[
	name=axis6,
	at={($(axis5.south east)+(0,-0.8cm)$)},anchor=north east,
       width=8cm,
       height=3cm,	
	axis y line=middle,
	axis x line=bottom,
	ymin=0,ymax=1.2,
	xmin=-3,xmax=4,	
	xlabel=$\tau$,
	ylabel= $h(t-\tau)$,	
	xtick={1.5, 3.5, 0},
	xticklabels={$t-2T$, $t$, 0},
	ytick={1},
	yticklabels={$2T$},
	every axis x label/.style={at={(ticklabel* cs:1.05)},    anchor=west},
	every axis y label/.style={at={(ticklabel* cs:1.05)},    anchor=west},	
	clip=false,
]
	\addplot [dashed, EntcBlue] coordinates {(-3,0) (0,0) (0,1) (1,1) (1,0) (4,0)};
	\addplot [thick, EntcRed] coordinates {(-3,0) (1.5,0) (1.5,1) (3.5, 0) (4,0)};
	\node at (axis cs:3,0.8) {$t>3T$};	
\end{axis}
\end{tikzpicture} 
\end{frame}