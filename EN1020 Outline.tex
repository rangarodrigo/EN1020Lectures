\documentclass[11pt, a4paper]{article}
\usepackage[utf8]{inputenc}
\usepackage[english]{babel}
\usepackage[svgnames]{xcolor}
\usepackage[colorlinks=true,linkcolor=magenta,urlcolor=magenta]{hyperref}
\usepackage{fourier}
\usepackage{tikz}
\usepackage{textcomp}
\usepackage{multicol}
\usepackage{booktabs}
\usepackage{fancyhdr}



\usepackage[margin=0.75in]{geometry}

\usepackage{array}
\newcolumntype{L}[1]{>{\raggedright\let\newline\\\arraybackslash\hspace{0pt}}m{#1}}
\newcolumntype{C}[1]{>{\centering\let\newline\\\arraybackslash\hspace{0pt}}m{#1}}
\newcolumntype{R}[1]{>{\raggedleft\let\newline\\\arraybackslash\hspace{0pt}}m{#1}}

\newcommand{\titlecolor}{\color{SlateBlue} \usefont{OT1}{lmss}{m}{n}}
\newcommand{\sectioncolor}{\color{SteelBlue} \usefont{OT1}{lmss}{m}{n}}

%\renewcommand{\refname}{}
\renewcommand{\arraystretch}{1.2}

\title{\Large Department of Electronic and Telecommunication Engineering\\The University of Moratuwa, Sri Lanka\\{\LARGE \bf \titlecolor \textsc{EN1020 Circuits, Signals, and Systems}}\\
{\large Course Outline---January 2024}}
\date{\vspace{-0.5in}}



\begin{document}

\maketitle

\noindent \tikz \draw (0,0) -- (\textwidth,0);

\section{\sectioncolor Introduction}
Signals and systems find many application in communications, automatic control, and form the basis for signal processing, communication, machine vision, and pattern recognition.   Electrical signals (voltages and currents in circuits, electromagnetic communication signals), acoustic signals, image and video signals, and biological signals are all example of signals that we encounter. They are functions of independent variables and carry information. We define a system as a mathematical relationship between an input signal and an output signal. We can use systems to analyze and modify signals. We can realize any systems using electrical and electronic  circuits (in addition to other systems like mechanical systems): The theoretical analysis that we do manifests in such circuits.   Circuits, Signals, and systems have brought about revolutionary changes. In this course we will study the fundamentals of circuits, signals, and systems. Types of signals in continuous time and discrete time, linear time-invariant (LTI) systems, Fourier series analysis  and realizing these systems in circuits are the core components of the course.

Modules such as Signals and Systems, Digital Signal Processing, and many modules in the telecommunications pathway build on the fundamental knowledge that we would gain from this course module.

\section{\sectioncolor Learning Outcomes}
After completing this course you will be able to do the following:
\begin{itemize}
    \item Explain the fundamental tools in  electrical circuit analysis.
    \item Apply network theorems in analysing electrical circuits.
    \item Differentiate between continuous-time, discrete-time and digital signals, and techniques applicable to the analysis of each type.
    \item Use Fourier series techniques to understand frequency domain characteristics of signals.
    \item Apply appropriate theoretical principles to characterize the behaviour of Linear Time Invariant (LTI) systems.
\end{itemize}

\section{\sectioncolor Contents}
%\begin{multicols}{2}
\begin{enumerate}
  \item Circuit Theory 
  \begin{itemize}
    \item[] Circuit vs. wavelength, circuit as a graph/network. Charge, current, voltage, power, and energy. units of measurement. LTI resistor, capacitor, and inductor. KCL and KVL. Ideal current and voltage sources, dependent sources, device modelling, RLC transient solutions using differential equations, concepts of transients vs. steady state. Resonance, mutual inductance, electromagnetic coupling, and analysis. Transformer as a coupled element.
  \end{itemize}
  \item Circuit Analysis Using Network Theorems
  \begin{itemize}
    \item[] Ground as a node, nodal analysis, Y matrix, node voltage and stimulus vector, super nodes, mesh analysis. Network theorems: superposition, Thevenin’s, Norton’s, Millman’s. Source transformation and network equivalence, source transportation, substitution theorem, maximum power transfer, $\mathrm{Y}-\Delta$ transformation. Two-port theory: impedance, admittance, hybrid, and ABCD parameters.
  \end{itemize}
  \item Introduction to Signals and Systems 
  \begin{itemize}
    \item[] Classification of signals as continuous-time, discrete-time and digital. Introduction to impulse and step functions. Introduction to systems and input-output relationships. Simple classes of signals such as sinusoid and exponential signals. Characterizing Linear Time-Invariant (LTI) systems. Overview of the analysis techniques applicable to each type of signal/system and their interrelationships.
  \end{itemize}
  \item Linear Time-Invariant (LTI) Systems
  \begin{itemize}
    \item[] Characteristics of LTI systems. Characterizing the input-output relationship of continuous- and discrete-time LTI systems in the time domain. The convolution theorem and its application to LTI systems. RLC circuit an LTI system.
  \end{itemize}
  \item Fourier Series
  \begin{itemize}
    \item[] Overview of Fourier analysis as the representation of signals with complex sinusoids. The Fourier series representation of periodic signals. Properties of the Fourier series. Characterizing LTI systems in the frequency domain. Introduction to Fourier transform.
  \end{itemize}
\end{enumerate}
%\end{multicols}
\section{Prerequisites}
Calculus.


\section{Contact Hours, Course Material, Etc.}
\begin{tabbing}
  \hspace{2in}\= Ranga Rodrigo \kill
  % \> for next tab, \\ for new line...
  Instructors: \>  Dr. Ranga Rodrigo.\\
  \> Electronics Building, Room 111.\\
                \> ranga@uom.lk, 011 264 0422.\\
                \\
                 \>  Dr. Upeka Premaratne.\\
  \> Electronics Building\\
                \> upeka@uom.lk.\\
                \\
  Lectures: \> 2 hours per week: Tuesdays 8:15 am. to 10:15 am.\\
  Tutorials: \> Every Wednesday from 3:15 pm. to 5:15 pm. (This will be notified in advance.)\\
  Labs: \> As scheduled in EN1094.\\
  \\
  Office hours (drop in): \> Please call me to set up an online appointment due to the current situation.\\
  \> Set up an appointment if you wish to meet outside office hours.\\
  \\
  %Website: \> \href{http://www.ent.mrt.ac.lk/~ranga/courses/en4620_2010.html}{http://www.ent.mrt.ac.lk/\texttildelow %ranga/courses/en4620\_2010.html} and\\
  Moodle page \>
  \href{https://online.uom.lk/course/view.php?id=23094}{https://online.uom.lk/course/view.php?id=23094}

\end{tabbing} 

\section{Evaluation Scheme}
%\begin{table}[h!]
\noindent
\begin{tabular}{@{}lllr@{}}
  \toprule
  Item   & Date& Weight& Minimum\\
  \midrule
  In-class quizzes (5 out of 8) & Surprise & 20\% & 50\%\\

  Mid-semester examination  &  To be decided& 20\% & 50\%\\

  Final examination & to be scheduled & 60\% & 50\%\\
  \bottomrule
\end{tabular}
%\end{table}

\section{Schedule}
%
%
%\begin{tabular}{@{}llp{2in}p{3in}@{}}
%\toprule
%Event 	&	Date 	&	Description	&						Material	\\
%\midrule
%Lecture 1	&	15-Dec	&	Introduction to signals and systems	&	\href{https://github.com/rangarodrigo/EN1020Lectures/blob/master/a\%20Signals\%20and\%20Systems\%20Introduction.pdf}{	a Signals and Systems Introduction}, \par	Oppenheim 1.0, 1.1, 1.2	\\
%Lecture 2	&	16-Dec	&	Signals	&	\href{https://github.com/rangarodrigo/EN1020Lectures/blob/master/b\%20Signals\%20and\%20Systems\%20Signals.pdf}{	b Signals and Systems Signals}, \par	Oppenheim 1.3, 1.4, 1.5, 1.6	\\
%Lecture 3	&	17-Dec	&	Continuous-time Fourier series	&	\href{https://github.com/rangarodrigo/EN1020Lectures/blob/master/c\%20Signals\%20and\%20Systems\%20Fourier\%20Series.pdf	}{c Signals and Systems Fourier Series}, \par	Oppenheim 3.0, 3.1, 3.3	\\
%Lecture 4	&	18-Dec	&	Continuous-time Fourier series properties	&Oppenheim 3.5	\\
%Lecture 7	&	21-Dec	&	Linear time-invariant systems	&	\href{https://github.com/rangarodrigo/EN1020Lectures/blob/master/f\%20Signals\%20and\%20Systems\%20Linear\%20Time\%20Invariant\%20Systems.pdf	}{f Signals and Systems Linear Time Invariant Systems}, \par	Oppenheim 2.0	\\
%Lecture 8	&	22-Dec	&	Convolution	&	 \par	Oppenheim 2.1, 2.2, 3.3	\\
%Lecture 9	&	23-Dec	&	Properties of LTI systems	&	 \par	Oppenheim 2.3	\\
%%Lecture 10	&	24-Dec	&	Discrete-time Fourier series	&	\href{https://github.com/rangarodrigo/EN1060Lectures/blob/master/g\%20Signals\%20and\%20Systems\%20Discrete\%20Time\%20Fourier\%20Series.pdf	}{g Signals and Systems Discrete Time Fourier Series}, \par	Oppenheim 3.6	\\
%%Lecture 11	&	25-Dec	&	Discrete-time Fourier transform	&	\href{https://github.com/rangarodrigo/EN1060Lectures/blob/master/h\%20Signals\%20and\%20Systems\%20Discrete\%20Time\%20Fourier\%20Transform.pdf	}{h Signals and Systems Discrete Time Fourier Transform}, \par	Oppenheim 5.0, 5.1	\\
%%Lecture 12	&	26-Dec	&	The Laplace transform	&	\href{https://github.com/rangarodrigo/EN1060Lectures/blob/master/i\%20Signals\%20and\%20Systems\%20Laplace\%20Transforms.pdf	}{i Signals and Systems Laplace Transforms}, \par	Oppenheim 9.0, 9.1, 9.2, 9.3, 9.4, 9.5, 9.6, 9.7	\\
%%Lecture 13	&	27-Dec	&	Systems with Laplace transform, z Transform	&	\href{https://github.com/rangarodrigo/EN1060Lectures/blob/master/j\%20Signals\%20and\%20Systems\%20z\%20Transfroms.pdf	}{j Signals and Systems z Transfroms}, \par	Oppenheim 9.7,  10.0, 10.1, 10.4, 10.5, 10.6, 10.7	\\
%%Lecture 14	&	28-Dec	&	Systems with z Transform, Sampling and reconstruction	&	\href{https://github.com/rangarodrigo/EN1060Lectures/blob/master/k\%20Signals\%20and\%20Systems\%20Sampling.pdf}{	k Signals and Systems Sampling}, \par	Oppenheim 10.6, 10.7 , 7.0, 7.1	\\
%\bottomrule
%\end{tabular}
%
%\vspace{0.5in}

%\section{Text Books}
\nocite{OPPENH97}
\nocite{HSUHWE95}

\bibliographystyle{IEEEtran}
\bibliography{../../../../Research/ranga_bib/ranga_bib}
\end{document}
